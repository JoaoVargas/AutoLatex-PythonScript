\documentclass[12pt , a4paper]{article}
\usepackage[utf8]{inputenc}
\usepackage[brazilian]{babel}
\usepackage{amsmath}
\usepackage{amsfonts}
\usepackage{amssymb}
\usepackage[T1]{fontenc}
\usepackage{graphicx}
\usepackage{physics}
\usepackage{siunitx}
\usepackage{minted}
\usepackage{scrextend}
\usepackage{lipsum}
\usepackage{changepage}
\setminted[python]{breaklines, framesep=2mm, fontsize=\footnotesize, numbersep=5pt}
\input{commands}

\begin{document}

%%--CABEÇALHO--%%
\begin{center}
    {\huge Lista 1 - Exercícios \par}
    {\LARGE Python \par}
\end{center}

%%--Problema 1--%%
\problem Maior3\\ 
Recebe três valores, e retorna o maior dos três.
\begin{adjustwidth}{1cm}{}
Argumentos:
\begin{adjustwidth}{1cm}{}a (float): primeiro valor;\\
b (float): segundo valor;\\
c (float): terceiro valor;
\end{adjustwidth}
Argumentos:
\begin{adjustwidth}{1cm}{}float: o maior entre os três valores.
\end{adjustwidth}
\end{adjustwidth}

%%--Problema 2--%%
\problem Menor3\\ 
Recebe três valores, e retorna o menor dos três.
\begin{adjustwidth}{1cm}{}
Argumentos:
\begin{adjustwidth}{1cm}{}a (float): primeiro valor;\\
b (float): segundo valor;\\
c (float): terceiro valor;
\end{adjustwidth}
Argumentos:
\begin{adjustwidth}{1cm}{}float: o menor entre os três valores.
\end{adjustwidth}
\end{adjustwidth}

%%--Problema 3--%%
\problem Ano bissexto\\ 
Receba os três lados de um triângulo. Informe se os valores podem ser um triângulo. Indique, caso os lados formem um triângulo, se o mesmo é: equilátero, isósceles ou escaleno.
\begin{adjustwidth}{1cm}{}
Argumentos:
\begin{adjustwidth}{1cm}{}a (float): primeiro lado;\\
b (float): segundo lado;\\
c (float): terceiro lado;
\end{adjustwidth}
Argumentos:
\begin{adjustwidth}{1cm}{}string: um texto indicando o resultado,   conforme aparece nos testes no final desse arquivo.
\end{adjustwidth}
\end{adjustwidth}

%%--Problema 4--%%
\problem Maior dia do mes\\ 
Determine se um ano é bissexto ou não.
\begin{adjustwidth}{1cm}{}
Argumentos:
\begin{adjustwidth}{1cm}{}ano (int): um ano, no formato de 4 dígitos.
\end{adjustwidth}
Argumentos:
\begin{adjustwidth}{1cm}{}bool: True ou False (verdadeiro ou falso), caso a ano seja ou não bissexto.
\end{adjustwidth}
\end{adjustwidth}

%%--Problema 5--%%
\problem Data valida\\ 
Retorna o último dia do mês para um determinado ano e mês.\\
Os valores possíveis são: 28, 29, 30 ou 31.\\
Devem ser considerados os anos bissextos.\\
Uma função separada para determinar se um ano é bissextopoderia ser utilizada.
\begin{adjustwidth}{1cm}{}
Argumentos:
\begin{adjustwidth}{1cm}{}mes (int): um mês no formato de dois dígitos;\\
ano (int): um ano, no fomato de 4 dígitos;
\end{adjustwidth}
Argumentos:
\begin{adjustwidth}{1cm}{}int: um inteiro indicando o último dia válido para aquele mês e ano.
\end{adjustwidth}
\end{adjustwidth}

%%--Problema 6--%%
\problem Baskara\\ 
Recebe uma string no formato dd/mm/aaaa e informaum valor lógico indicando se a data é válida ou não.\\
Verifica se ano é bissexto e outros detalhes.\\
Confira os testes no final do arquivo para mais detalhes.
\begin{adjustwidth}{1cm}{}
Argumentos:
\begin{adjustwidth}{1cm}{}data (string): data no formato "dd/mm/aaaa".
\end{adjustwidth}
Argumentos:
\begin{adjustwidth}{1cm}{}bool: True ou False, indicando se a datá é válida ou não.
\end{adjustwidth}
\end{adjustwidth}

%%--Problema 7--%%
\problem Acrescimo nota bb\\ 
Calcule as raízes de uma equação do segundo grau, na forma ax2 + bx + c. A função recebe a, b e c e faz as consistências, informando ao usuário nas seguintes situações: - Se o usuário informar o valor de A igual a zero é uma equaçao do 1o grau.\\
 - Se o delta calculado for negativo, a equação não possui raizes reais.\\
 Devolva uma tupla vazia.\\
 - Se o delta calculado for igual a zero a equação possui apenas uma raiz real. Devolva uma tupla com um único valor.\\
 - Se o delta for positivo, a equação possui duas raiz reais.\\
 Devolva uma tupla com dois elementos.
\begin{adjustwidth}{1cm}{}
Argumentos:
\begin{adjustwidth}{1cm}{}a (float): valor a da equação;\\
b (float): valor b da equação;\\
c (float): valor c da equação;
\end{adjustwidth}
Argumentos:
\begin{adjustwidth}{1cm}{}tupla de floats: uma tupla, contando os valores das raízes, sendouma raiz, duas raízes ou uma tupla vazia caso não existam raízes.
\end{adjustwidth}
\end{adjustwidth}



\end{document}